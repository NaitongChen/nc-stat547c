% !TEX root = ../main.tex

% Body section
\section{Body}
First we set up the notaions and assumptions that are used throughout the report.
\subsection{Notations and Assumptions}
Let $X_1,\cdots,X_n$ be \iid continuous random variables in $\bbR$ with probability density function $f(x;\theta_0)$, where $\theta_0 \in \Theta$ is a unknown parameter.\\\\
Write the likelihood function
\begin{align}
L(\theta) = \prod_{i=1}^n f(x_i;\theta),
\end{align}
and subsequently the log-likelihood function
\begin{align}
l(\theta) = \sum_{i=1}^n \log f(x_i;\theta).
\end{align}
Then the maximum likelihood estimator can be computed as
\begin{align}
\hat{\theta} = \argmax_{\theta\in\Theta}l(\theta) = \argmax_{\theta\in\Theta}L(\theta).
\end{align}
Note then since \bred conditons \ered, a necessary consequence is that
\begin{align}
\left.\frac{\partial l(\theta)}{\partial \theta} \right\vert_{\theta=\theta_0} = 0.
\end{align}
We use $\bbE_{\theta}[X]$ to denote expectation with respect to $X$ under $f(x;\theta)$. Using this notation we define the Fisher information
\begin{align}
I(\theta) = \bbE_{\theta}\left[ \left( \frac{\partial}{\partial\theta}\log f(X;\theta) \right)^2 \right]. \label{eqn:fisher}
\end{align}
Throughout the report, we assume the following regularity conditions.
\begin{itemize}
\item (R1) $f(x;\theta)$ is identifiable: $\theta_1\neq\theta_2\implies f(x;\theta_1)\neq f(x;\theta_2)$.
\item (R2) $f(x;\theta)$ has common support for all $\theta\in\Theta$.
\item (R3) $\theta_0$ is an iterior point in $\Theta$.
\item (R4) $f(x;\theta)$ is twice differentiable in $\theta$.
\item (R5) The integral $\int f(x;\theta)$ can be different twice in $\theta$ under the integral sign.
\end{itemize}

\subsection{Intermediate Results}
We prove two lemmas that are essential to developing the main results of this report.\\\\
\textbf{Lemma 1}: Under regularity conditions, \bred (define $P_{\theta_0}$), \ered
\begin{align}
\lim_{n\to\infty} P_{\theta_0}\left( L(\theta_0 \mid X) > L(\theta \mid X) \right) = 1, \forall \theta \neq \theta_0.
\end{align}
\begin{proof}
We begin by taking the log on both sides of the inequality on the LHS and rearrange. 
\begin{align}
L(\theta_0 \mid X) > L(\theta \mid X) &\implies \sum_{i=1}^n \log f(X_i;\theta_0) > \sum_{i=1}^n \log f(X_i;\theta)\\
&\implies \sum_{i=1}^n \left(\log f(X_i;\theta_0) - \log f(X_i;\theta)\right) < 0\\
&\implies \frac{1}{n}\sum_{i=1}^n\log\left( \frac{f(X_i;\theta)}{f(X_i;\theta_0)} \right) < 0. \label{eqn:rearrange}
\end{align}
Since $X_i$'s are \iid, the summands are independent, and so by the Weak Law of Large Numbers,
\begin{align}
\frac{1}{n}\sum_{i=1}^n\log\left( \frac{f(X_i;\theta)}{f(X_i;\theta_0)} \right) &\convP \bbE_{\theta_0}\left[ \log \left( \frac{f(X_1; \theta)}{f(X_1; \theta_0)} \right) \right].
\end{align}
Note that $-\log(\cdot)$ is strictly convex, then by Jensen's inequality, we can establish, on the RHS of the above equation,
\begin{align}
-\bbE_{\theta_0}\left[ \log \left( \frac{f(X_1; \theta)}{f(X_1; \theta_0)} \right) \right] = \bbE_{\theta_0}\left[ -\log \left( \frac{f(X_1; \theta)}{f(X_1; \theta_0)} \right) \right] > -\log \bbE_{\theta_0}\left[ \frac{f(X_1; \theta)}{f(X_1; \theta_0)}  \right].
\end{align}
Now note
\begin{align}
\log \bbE_{\theta_0} \left[ \frac{f(X_1; \theta)}{f(X_1; \theta_0)} \right] &= \log \int_\bbR \frac{f(x; \theta)}{f(x; \theta_0)} f(x; \theta_0) dx = \log\int_\bbR \frac{f(x; \theta)} dx = \log 1 = 0.
\end{align}
Together, we have
\begin{align}
\frac{1}{n}\sum_{i=1}^n\log\left( \frac{f(X_i;\theta)}{f(X_i;\theta_0)} \right) &\convP \bbE_{\theta_0}\left[ \log \left( \frac{f(X_1; \theta)}{f(X_1; \theta_0)} \right) \right] < 0. \label{eqn:convp}
\end{align}
To show the desired equation, it is equivalent to show, by \cref{eqn:rearrange},
\begin{align}
\lim_{n\to\infty}P_{\theta_0}\left( \frac{1}{n}\sum_{i=1}^n\log\left( \frac{f(X_i;\theta)}{f(X_i;\theta_0)} \right) < 0 \right) = 1.
\end{align}
By \cref{eqn:convp}, we know that for all $\epsilon > 0$,
\begin{align}
\lim_{n\to\infty}P_{\theta_0}\left( \left| \frac{1}{n}\sum_{i=1}^n\log\left( \frac{f(X_i;\theta)}{f(X_i;\theta_0)} \right) - \bbE_{\theta_0}\left[ \log \left( \frac{f(X_1; \theta)}{f(X_1; \theta_0)} \right) \right] \right| < \epsilon \right) = 1.
\end{align}
Again by rearranging the inequality inside, we get
\begin{align}
\bbE_{\theta_0}\left[ \log \left( \frac{f(X_1; \theta)}{f(X_1; \theta_0)} \right) \right] - \epsilon < \frac{1}{n}\sum_{i=1}^n\log\left( \frac{f(X_i;\theta)}{f(X_i;\theta_0)} \right) < \bbE_{\theta_0}\left[ \log \left( \frac{f(X_1; \theta)}{f(X_1; \theta_0)} \right) \right] + \epsilon. \label{eqn:long}
\end{align}
Note that the probability of event \cref{eqn:long} is less than or equal to that of
\begin{align}
\frac{1}{n}\sum_{i=1}^n\log\left( \frac{f(X_i;\theta)}{f(X_i;\theta_0)} \right) < \bbE_{\theta_0}\left[ \log \left( \frac{f(X_1; \theta)}{f(X_1; \theta_0)} \right) \right] + \epsilon.
\end{align}
Since $\bbE_{\theta_0}\left[ \log \left( \frac{f(X_1; \theta)}{f(X_1; \theta_0)} \right) \right] < 0$, by fixing $\epsilon = -\bbE_{\theta_0}\left[ \log \left( \frac{f(X_1; \theta)}{f(X_1; \theta_0)} \right) \right] > 0$, we have
\begin{align}
&\lim_{n\to\infty}P_{\theta_0}\left( \left| \frac{1}{n}\sum_{i=1}^n\log\left( \frac{f(X_i;\theta)}{f(X_i;\theta_0)} \right) - \bbE_{\theta_0}\left[ \log \left( \frac{f(X_1; \theta)}{f(X_1; \theta_0)} \right) \right] \right| < \epsilon \right)\\
&\leq \lim_{n\to\infty}P_{\theta_0}\left( \frac{1}{n}\sum_{i=1}^n\log\left( \frac{f(X_i;\theta)}{f(X_i;\theta_0)} \right) < \bbE_{\theta_0}\left[ \log \left( \frac{f(X_1; \theta)}{f(X_1; \theta_0)} \right) \right] + \epsilon \right)\\
&= \lim_{n\to\infty}P_{\theta_0}\left( \frac{1}{n}\sum_{i=1}^n\log\left( \frac{f(X_i;\theta)}{f(X_i;\theta_0)} \right) < 0 \right) = 1.
\end{align}
Therefore, we conclude that
\begin{align}
\lim_{n\to\infty} P_{\theta_0}\left( L(\theta_0 \mid X) > L(\theta \mid X) \right) = 1, \forall \theta \neq \theta_0.
\end{align}
\end{proof}
$ $\\
\textbf{Lemma 2}: Under regularity conditions, 
\begin{align}
\bbE\left( \frac{\partial\log f(X;\theta)}{\partial \theta} \right) = 0, \text{ and }
I(\theta) = -\bbE_\theta\left[ \frac{\partial^2\log f(X;\theta)}{\partial \theta^2} \right] = \var\left( \frac{\partial\log f(X;\theta)}{\partial \theta} \right).
\end{align}
\begin{proof}
We begin with differentiating both sides of $1 = \int_\bbR f(x;\theta) dx$.
\begin{align}
0 &= \frac{\partial}{\partial\theta}\int_\bbR f(x;\theta)dx\\
&= \int_\bbR \frac{\partial}{\partial\theta} f(x;\theta)dx\\
&= \int_\bbR \frac{\frac{\partial}{\partial\theta} f(x;\theta)}{f(x;\theta)}f(x;\theta) dx\\
&= \int_\bbR \frac{\partial \log f(x;\theta)}{\partial\theta}f(x;\theta) dx \label{eqn:trick_1} \\
&= \bbE_\theta\left[ \frac{\partial \log f(x;\theta)}{\partial\theta} \right].
\end{align}
Differentiating with respect to $\theta$ again, we get
\begin{align}
0 &= \frac{\partial}{\partial\theta}\int_\bbR \frac{\partial \log f(x;\theta)}{\partial\theta}f(x;\theta) dx\\
&= \int_\bbR \frac{\partial^2 \log f(x;\theta)}{\partial\theta^2}f(x;\theta) dx + \int_\bbR \frac{\partial \log f(x;\theta)}{\partial\theta}\frac{\partial f(x;\theta)}{\partial\theta} dx\\
&= \int_\bbR \frac{\partial^2 \log f(x;\theta)}{\partial\theta^2}f(x;\theta) dx + \int_\bbR \frac{\partial \log f(x;\theta)}{\partial\theta}\frac{\partial \log f(x;\theta)}{\partial\theta}f(x;\theta) dx \label{eqn:trick_2} \\
&= \bbE_\theta\left[ \frac{\partial^2 \log f(x;\theta)}{\partial\theta^2} \right] + \bbE_\theta\left[ \left(\frac{\partial \log f(x;\theta)}{\partial\theta}\right)^2 \right],
\end{align}
where \cref{eqn:trick_2} used the same trick as \cref{eqn:trick_1}. By \cref{eqn:fisher}, we have
\begin{align}
I(\theta) = \bbE_\theta\left[ \left(\frac{\partial \log f(x;\theta)}{\partial\theta}\right)^2 \right] = -\bbE_\theta\left[ \frac{\partial^2 \log f(x;\theta)}{\partial\theta^2} \right].
\end{align}
Finally, since $\bbE_\theta\left[ \frac{\partial \log f(x;\theta)}{\partial\theta} \right] = 0$,
\begin{align}
\var\left( \frac{\partial \log f(x;\theta)}{\partial\theta} \right) = \bbE_\theta\left[ \left(\frac{\partial \log f(x;\theta)}{\partial\theta}\right)^2 \right] - \left( \bbE_\theta\left[ \frac{\partial \log f(x;\theta)}{\partial\theta} \right] \right)^2 = I(\theta).
\end{align}
Together, we conclude that
\begin{align}
\bbE\left( \frac{\partial\log f(X;\theta)}{\partial \theta} \right) = 0, \text{ and }
I(\theta) = -\bbE_\theta\left[ \frac{\partial^2\log f(X;\theta)}{\partial \theta^2} \right] = \var\left( \frac{\partial\log f(X;\theta)}{\partial \theta} \right).
\end{align}
\end{proof}

\subsection{Asymptotic Normality of MLE}
We first establish the consistency of MLE.\\\\
\textbf{Lemma 3}: Under regularity conditions, $\hat{\theta}$ is a consistent estimator of $\theta_0$.
\begin{proof}
By dropping the condition that $\theta\neq\theta_0$, Lemma 1 states that
\begin{align}
\lim_{n\to\infty}P_{\theta_0}\left( L(\theta_0 \mid X) \geq L(\theta \mid X) \right) = 1. \label{eqn:geq}
\end{align}
And by definition of $\hat{\theta}_n$, \bred (explain n and X) \ered,
\begin{align}
L(\hat{\theta}_n \mid X) \geq L(\theta_0 \mid X), \forall X. \label{eqn:leq}
\end{align}
By letting $\theta = \hat{\theta}_n$ for each $n$, we have by \cref{eqn:leq},
\begin{align}
\lim_{n\to\infty}P_{\theta_0}\left( L(\theta_0 \mid X) \leq L(\hat{\theta}_n \mid X) \right) = 1,
\end{align}
and by \cref{eqn:geq},
\begin{align}
\lim_{n\to\infty}P_{\theta_0}\left( L(\theta_0 \mid X) \geq L(\hat{\theta}_n \mid X) \right) = 1.
\end{align}
Together, 
\begin{align}
\lim_{n\to\infty}P_{\theta_0}\left( L(\theta_0 \mid X) = L(\hat{\theta}_n \mid X) \right) = 1. \label{eqn:equal}
\end{align}
Lemma 1 implies $\theta_0$ is the unique maximizer of $L(\theta|X)$ under $P_{\theta_0}$ as $n$ approaches infinity, so \cref{eqn:equal} implies 
\begin{align}
\lim_{n\to\infty}P_{\theta_0}(\theta_0 = \hat{\theta}_n) = 1 \implies \lim_{n\to\infty}P_{\theta_0}\left( \left| \hat{\theta}_n - \theta_0 \right| > \epsilon \right) = 0, \forall \epsilon > 0.
\end{align}
We then have that $\hat{\theta}$ is a consistent estimator of $\theta_0$.
\end{proof}
$ $\\
We now use this fact to prove the asymptotic normality of MLE.\\\\
\textbf{Theorem 1}: Under regularity conditions, 
\begin{align}
\sqrt{n}(\hat{\theta} - \theta) \convD \calN\left(0, \left[ I(\theta_0) \right]^{-1}\right),
\end{align}
where $I(\theta_0) = \bbE_{\theta_0}\left[ \left( \left.\frac{\partial \log f(X;\theta)}{\partial\theta}\right\vert_{\theta=\theta_0} \right)^2 \right]$.
\begin{proof}
By the Mean Value Theorem, for $f:\bbR \rightarrow \bbR$ continuous on $[a,b]$ and differentiable on $(a,b)$, for all $c\in(a,b)$,
\begin{align}
\frac{f(a)-f(b)}{a-b} = f'(c).
\end{align}
Let $f(\theta) = l'(\theta) = \frac{\partial l(\theta)}{\partial \theta}, a=\hat{\theta}, b=\theta_0, c=\theta_1 \in (\theta_0, \hat{\theta})$. Then with $l''(\theta) = \frac{\partial^2 l(\theta)}{\partial \theta^2}$, the above equation becomes
\begin{align}
\frac{l'(\hat{\theta}) - l'(\theta_0)}{\hat{\theta} - \theta_0} = l''(\theta_1).
\end{align}
We know $l'(\hat{\theta}) = 0$, then the equation above becomes
\begin{align}
0 = l'(\theta_0) + (\hat{\theta} - \theta_0)l''(\theta_1) \implies \sqrt{n}(\hat{\theta} - \theta_0) = -\frac{\sqrt{n}l'(\theta_0)}{l''(\theta_1)}. \label{eqn:frac}
\end{align}
We first look at the denominator of \cref{eqn:frac}. By Lemma 3, $\hat{\theta} \convP \theta_0$. Then since $\theta_1\in(\theta_0, \hat{\theta})$, we must have $\theta_1 \convP \theta_0$. Then by Proposition 10.7 from the lecture notes, 
\begin{align}
l''(\theta_1) \convP l''(\theta_0).
\end{align}
Now by the Weak Law of Large Numbers and Lemma 2,
\begin{align}
l''(\theta_0) = \frac{1}{n}\sum_{i=1}^n \left.\frac{\partial^2}{\partial \theta^2}\log f(x_i; \theta)\right\vert_{\theta=\theta_0} \convP \bbE_{\theta_0}\left[ \left.\frac{\partial^2}{\partial \theta^2}\log f(X_1; \theta)\right\vert_{\theta=\theta_0} \right] = -I(\theta_0). \label{eqn:denom}
\end{align}
Now we look at the numerator of \cref{eqn:frac}. By Lemma 2, $\bbE_{\theta_0}\left[ \left.\frac{\partial}{\partial\theta}\log f(X_1;\theta)\right\vert_{\theta=\theta_0} \right] = 0$ and $I(\theta_0) = \var\left( \left.\frac{\partial}{\partial\theta}\log f(X_1;\theta)\right\vert_{\theta=\theta_0} \right)$. Then by the Central Limit Theorem,
\begin{align}
\sqrt{n}l'(\theta_0) &= \sqrt{n}\frac{1}{n}\sum_{i=1}^n \left.\frac{\partial}{\partial\theta}\log f(x_i;\theta)\right\vert_{\theta=\theta_0}\\
&= \sqrt{n}\left( \frac{1}{n}\sum_{i=1}^n \left.\frac{\partial}{\partial\theta}\log f(x_i;\theta)\right\vert_{\theta=\theta_0} - \bbE_{\theta_0}\left[ \left.\frac{\partial}{\partial\theta}\log f(X_1;\theta)\right\vert_{\theta=\theta_0} \right] \right)\\
&\convD \calN\left(0, \var\left( \left.\frac{\partial}{\partial\theta}\log f(X_1;\theta)\right\vert_{\theta=\theta_0} \right)\right)\\
&= \calN\left( 0, I(\theta_0) \right). \label{eqn:numerator}
\end{align}
Together by \cref{eqn:frac,eqn:denom,eqn:numerator} and Slutsky's Theorem,
\begin{align}
\sqrt{n}(\hat{\theta} - \theta_0) = -\frac{\sqrt{n}l'(\theta_0)}{l''(\theta_1)} \convD \frac{\calN\left(0,I(\theta_0)\right)}{I(\theta_0)} = \calN\left(0, \left[ I(\theta_0) \right]^{-1}\right).
\end{align}
Therefore we conclude that $\sqrt{n}(\hat{\theta} - \theta_0)\convD\calN\left(0, \left[ I(\theta_0) \right]^{-1}\right)$.
\end{proof}
$ $\\
Out of all possible estimators for $\theta_0$, how good is the MLE? The next result gives us some insight into the quality of MLE compared to others in terms of the variance of these estimators.\\\\
\textbf{Theorem 2}: Let $Y = u(X_1,\cdots,X_n)$ be an unbiased estimator of $\theta_0$ such that $\bbE_{\theta_0}[Y] = \theta_0$. Then under regularity conditions, $\var(Y) \geq \frac{1}{nI(\theta_0)}$.
\begin{proof}
We first expand $E_{\theta_0}[Y]$.
\begin{align}
\theta_0 = E_{\theta_0}[Y] = \int_\bbR\cdots\int_\bbR u(x_1,\cdots,x_n)\prod_{i=1}^nf(x_i;\theta_0)dx_1\cdots dx_n.
\end{align}
Differentiating both sides with respect to $\theta_0$ gives
\begin{align}
1 &= \int_\bbR\cdots\int_\bbR u(x_1,\cdots,x_n) \left(\sum_{i=1}^n \frac{\partial f(x_i;\theta_0)}{\partial\theta_0} \frac{1}{f(x_i;\theta_0)}\right) \prod_{i=1}^nf(x_i;\theta_0)dx_1\cdots dx_n\\
&= \int_\bbR\cdots\int_\bbR u(x_1,\cdots,x_n) \left(\sum_{i=1}^n \frac{\partial \log f(x_i;\theta_0)}{\partial\theta_0} \right) \prod_{i=1}^nf(x_i;\theta_0)dx_1\cdots dx_n
\end{align}
By writing $Z = \sum_{i=1}^n \frac{\partial \log f(X_i;\theta_0)}{\partial\theta_0}$, we get, with $\rho$ denoting the correlation coefficient between $Y$ and $Z$,
\begin{align}
1 = \bbE_{\theta_0}[YZ] = \bbE_{\theta_0}[Y]\bbE_{\theta_0}[Z] + \rho\sqrt{\var(Y)}\sqrt{\var(Z)} \implies \rho = \frac{1}{\sqrt{\var(Y)}\sqrt{\var(Z)}}.
\end{align}
We note that since $X_i$'s are \iid,
\begin{align}
\var(Z) &= \var\left( \sum_{i=1}^n \frac{\partial \log f(X_i;\theta_0)}{\partial\theta_0} \right)\\
&= \sum_{i=1}^n \var\left(\frac{\partial \log f(X_i;\theta_0)}{\partial\theta_0}\right)\\
&= n\var\left(\frac{\partial \log f(X_1;\theta_0)}{\partial\theta_0}\right)\\
&= nI(\theta_0).
\end{align}
Then
\begin{align}
\rho = \frac{1}{\sqrt{nI(\theta_0)}\sqrt{\var(Y)}}.
\end{align}
By definition, $rho^2 \leq 1$, then
\begin{align}
\rho^2 = \frac{1}{nI(\theta_0)\var(Y)} \leq 1 \implies \var(Y) \geq \frac{1}{nI(\theta_0)}.
\end{align}
Therefore for any unbiased estimator $Y$ of $\theta_0$, we have $\var(Y) \geq \frac{1}{nI(\theta_0)}$.
\end{proof}
$ $\\
Note the lower bound on the variance is exactly the asymptotic variance of $\hat{\theta}$. This means that asymptotically, the MLE achieves the smallest possible variance out of all unbiased estimators of $\theta$. While this is a nice property, it remains rather theoretical. In practice, when we work with finitely many observations, it is not clear how far away we are from asymptotic normality. And existing heuristics from undergraduate statistics courses such as calling samples larger than $25$ or $30$ large enough is far from satisfying. Fortunately, \cite{anastasiou2015bounds} have done some pioneer work aimed at answering this exact question. We turn in the next section for a brief illustration of their result on the the closeness to the asymptotic normal distribution from the MLE approximated using finitely many observations.
% ...