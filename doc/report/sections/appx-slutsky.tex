% !TEX root = ../main.tex

% Exercises section

\section{Proof of Slutsky's Theorem}
\begin{theorem}
If $X_n \convD X$ and $Y_n \convP c$, then $X_nY_n \convD cX$.
\end{theorem}
\begin{proof}
We begin by noting that this proof follows closely to the lecture notes from STAT 560 by Professor Ruben Zamar.\\\\
We begin by writing $X_nY_n = X_n(Y_n-c)+cX_n$. Then, in order to achieve the final result, we need to show the following
\begin{itemize}
\item If $X_n \convD X$ and $Y_n \convP c$, then $X_n + Y_n \convD X + c$.
\item If $X_n \convD X$ and $Y_n \convP 0$, then $X_nY_n \convP 0$.
\item If $X_n \convD X$, then $cX_n \convD cX$.
\end{itemize}
If we have the above results, then 
\begin{align*}
Y_n - c \convP 0 \implies X_n(Y_n-c) \convP 0 \implies \implies X_n(Y_n-c) \convD 0 \implies X_nY_n = X_n(Y_n-c)+cX_n \convD 0 + cX = cX.
\end{align*}
\end{proof}
\begin{lemma}
If $X_n \convD X$ and $Y_n \convP c$, then $X_n + Y_n \convD X + c$.
\end{lemma}
\begin{proof}

\end{proof}
\begin{lemma}
If $X_n \convD X$ and $Y_n \convP 0$, then $X_nY_n \convP 0$.
\end{lemma}
\begin{proof}
By definition, for any $\epsilon>0$, given $\delta>0$, we can find $N$ such that $\forall n\geq N$, $X_nY_n\convP 0 \implies P(|X_nY_n|<\epsilon)>1-\delta$.\\\\
First, we find $K$ such that $K,-K$ are continuity points of $F_X$, the CDF of $X$, and
\[
P(|X|\leq K) = P(-K\leq X\leq K) = F_X(K) - F_X(-K) \geq 1-\frac{\delta}{4}.
\]
Then
\begin{align*}
P(|X_n|\leq K) &= P(-K\leq X_n \leq K)\\
&= F_{X_n}(K) - F_{X_n}(-K)\\
&= F_X(K) - F_X(-K) - (F_{X}(K) - F_{X_n}(K)) - (F_{X_n}(-K) - F_{X_n}(-K))
\end{align*}
\end{proof}
\begin{lemma}
If $X_n \convD X$, then $cX_n \convD cX$.
\end{lemma}
\begin{proof}

\end{proof}